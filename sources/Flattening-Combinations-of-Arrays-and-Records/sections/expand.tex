
\subsection{Expanding Records to Base Types}\label{sec:expand}

After record-specific syntax has been replaced by function applications, we must ensure that those and all other functions no longer contain any record types and record variables.
To achieve this we replace those record types and variables by the fields of those records instead.
We look at the \texttt{l2norm} function as an example:
%
\begin{lstlisting}
double
l2norm(struct Vector3 v)
{
    vx = vector3_get_x(v);
    vy = vector3_get_y(v);
    vz = vector3_get_z(v);
    return sqrt(vx * vx + vy * vy + vz * vz);
}
\end{lstlisting}
%
We aim to transform this function into one without any record types.
%
\begin{lstlisting}
double
l2norm(double x, double y, double z)
{
    vx = vector3_get_x(x, y, z);
    vy = vector3_get_y(x, y, z);
    vz = vector3_get_z(x, y, z);
    return sqrt(vx * vx + vy * vy + vz * vz);
}
\end{lstlisting}
%
In this example we see how the record argument \texttt{struct Vector3 v} is separated into three distinct arguments x, y, and z; corresponding to the three fields of the \texttt{Vector3} record.
Furthermore, all occurrences of this argument \texttt{v} are also replaced by the same three variables instead.

In the case where records and arrays are nested, this tranformation becomes non-trivial.
Consider the signature of the \texttt{timestep} function from the example.
Here the given record type is an array instead of a scalar value, furthermore it contains the nested \texttt{Vector3} record for the position, and an array \texttt{double[3]} for the velocity.
%
\begin{lstlisting}
struct Body[n]
timestep(struct Body[n] bodies, double dt)
\end{lstlisting}
%
After expanding the \texttt{Body} record, and the nested \texttt{Vector3} record, we expect the function signature to look like:
%
\begin{lstlisting}
double[n], double[n], double[n], /* pos  */
double[n,3],                     /* vel  */
double[n]                        /* mass */
timestep(double[n] x, double[n] y, double[n] z,
         double[n,3] vel, double[n] mass,
         double dt)
\end{lstlisting}
%
This example highlights multiple interesting cases.
Firstly, although the \texttt{mass} field of the body record is a scalar \texttt{double}, after expansion we expect it to become an array of type \texttt{double[n]} since \texttt{struct Body[n]} denotes an array of records instead of a scalar record, which consequently applies to all fields of the record.
Secondly, the body record contains a nested \texttt{Vector3} record, which itself is then expanded into its three distinct x, y, and z fields.
Similarly to the mass field, here we must also ensure that these fields become arrays of type \texttt{double[n]}.
Finally there is the velocity field, which itself is already an array of type \texttt{double[3]}.
In this case, the shape of the bodies array must be concatenated with the shape of the velocity field, resulting in the type \texttt{double[n,3]}.
