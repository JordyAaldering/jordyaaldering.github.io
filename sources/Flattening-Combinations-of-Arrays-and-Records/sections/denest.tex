
\subsection{Denesting Fields of Nested Records}

In order to be able to apply this transformation we need to have a mapping of record types to the fully denested fields and expanded shapes.
We call this mapping $\sigma$, and allow it to be indexed by a record type to get the expanded fields of that record.
In the case of the \texttt{Vector3} record, this would look as follows:
%
\begin{align*}
    \sigma[\texttt{Vector3}] = \makeenv{
        \envid{x} \texttt{double},\ 
        \envid{y} \texttt{double},\ 
        \envid{z} \texttt{double}
    }
\end{align*}
%
Whereas for the \texttt{Body[n]} array of records we expect the following:
%
\begin{align*}
    \sigma[\texttt{Body[n]}] = \makeenv{
        &\envid{x} \texttt{double[n]},\ 
        \envid{y} \texttt{double[n]},\ 
        \envid{z} \texttt{double[n]} \\
        &\envid{vel} \texttt{double[n,3]},\ 
        \envid{mass} \texttt{double[n]}
    }
\end{align*}
%
Note that this environment does not contain the \texttt{pos} field, and instead has already expanded that record into its nested x, y, and z fields.

% ----------------------------------------------------------------------

\subsubsection{Denesting Records}

We populate this environment using a function called \texttt{Denest\textsubscript{rt}}.
This function expects a record declaration as its first argument, and the (initially empty) accumulated environment $\sigma$ as its second argument.
All fields of this record are then denested separately using a function \texttt{Denest\textsubscript{f}}, whose resulting environments are combined to form the mapping $\sigma$ of the record type.
We apply this denesting function to all record types in a program, from top to bottom.
\[
    \texttt{Denest\textsubscript{rt}}\left(\vcenter{\ttfamily\setlength{\baselineskip}{1em}
        \hbox{{\color{blue}struct rt} \{}
        \hbox{\quad {\color{blue}type\textsubscript{1}[shp\textsubscript{1}]}: id\textsubscript{1};}
        \hbox{\quad \dots}
        \hbox{\quad {\color{blue}type\textsubscript{n}[shp\textsubscript{n}]}: id\textsubscript{n};}
        \hbox{\};}
    },\ \sigma\right)
    = \vcenter{
        \hbox{\quad \texttt{Denest\textsubscript{f}}(%
            \texttt{\color{blue}type\textsubscript{1}},\ 
            \texttt{\color{blue}shp\textsubscript{1}},\ 
            \texttt{id\textsubscript{1}},\ 
            \ensuremath{\sigma}%
        )}
        \hbox{\ensuremath{\cup}\, \dots}
        \hbox{\ensuremath{\cup} \texttt{Denest\textsubscript{f}}(%
            \texttt{\color{blue}type\textsubscript{n}},\ 
            \texttt{\color{blue}shp\textsubscript{n}},\ 
            \texttt{id\textsubscript{n}},\ 
            \ensuremath{\sigma}%
        )}
    }
\]
%
Because from this point on this record may be used in a nested fashion in all following record declarations, we add this record to the environment.

% ----------------------------------------------------------------------

\subsubsection{Denesting Fields}

The main body of work lies in the \texttt{Denest\textsubscript{f}} function.
Given a field name $id$ and its corresponding type and shape, this function computes the environment $\sigma'$ of that shape.
Additionally this function requires the thus far accumulated environment $\sigma$, which is required when looking up the environment of a previously defined record type in the case that $type$ is a record type.
We distinguish between base-type fields and record type fields.
In the case that we encounter a base-type field $id$, be it a scalar or an array, a mapping can immediately be added to the environment without additional work.
%
\begin{align*}
    \texttt{Denest\textsubscript{f}}(basetype,\ [\,],\ id,\ \sigma)
        &= \{\, id:\,basetype \,\} \\
    \texttt{Denest\textsubscript{f}}(basetype,\ shp,\ id,\ \sigma)
        &= \{\, id:\,basetype[shp] \,\}
\end{align*}
%
If instead $id$ is a scalar record type, we lookup the previously computed environment of that record type.
Since this field does not have a shape, there is nothing more to do and we can copy the environment as is.
Because we require that records are defined top-to-bottom, this environment must exist at this point.
Otherwise an incorrect program was provided and we can raise an error.
%
\[
    \texttt{Denest\textsubscript{f}}(recordtype,\ [\,],\ id,\ \sigma)
        = \sigma[recordtype]
\]
%
As we have seen in the example, we need to do some additional work in the case that $id$ is an array of a record type.
Not only does that record type need to be denested, but the shape of $id$ must be prepended to all fields of the denested record type as well, for which we use a new function: \texttt{prepend}.
%
\[
    \texttt{Denest\textsubscript{f}}(recordtype,\ shp,\ id,\ \sigma)
        = \texttt{prepend}(shp,\ \sigma[recordtype])
\]
%
Following is the \texttt{prepend} function.
Its first argument is the shape we want to prepend, and the second argument is the environment of the record type to which we want to prepend this shape.
For every field in this environment, we then prepend the given shape to the previous shape, resulting in a new environment that has the same identifiers and types as the given argument, but now with expanded shapes.
%
\begin{gather*}
    \texttt{prepend}(shp_{rt},\ \sigma_{rt})
    = \left\{\, \vcenter{
        \hbox{\ensuremath{id_1:\,type_1[shp_{rt} :: shp_1],}}
        \hbox{\ensuremath{\dots,}}
        \hbox{\ensuremath{id_n:\,type_n[shp_{rt} :: shp_n]}}
    } \,\right\}\\
    \text{where}\ \sigma_{rt} = \{\, id_1:\,type_1[shp_1],\ \dots,\ id_n:\,type_n[shp_n] \,\}
\end{gather*}
%
No case distinction is needed for scalar fields, since their shape is the empty list (as seen in Section~\ref{sec:sac}), and thus the concatenation $shp :: []$ will act as an identity on $shp$.
Additionally we do not need to worry about nested records at this point, as they have already been denested and thus at this point we only have base-types.

Using this environment we can now actually apply the transformation proposed in Section~\ref{sec:expand}.
Whenever we encounter a record type, a record argument, or a record identifier we replace it by the expanded base-types and identifiers accordingly.
