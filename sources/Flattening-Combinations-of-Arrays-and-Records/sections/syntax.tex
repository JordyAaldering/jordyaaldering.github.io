
\section{A Syntax for Records in \sac{}}\label{sec:syntax}

We try to match the syntax for records to that of C, following the overall syntactic
approach of \sac{}.
The required syntactic extensions are shown in Figure~\ref{fig:syntax}.

\begin{figure}[hbt]
\begin{verbatim}
<record-decl>     ::= 'struct' <id> '{' (<field-type> <id> ';')+ '}'
<field-type>      ::= <type> <static-shp>?
                   |  'struct' <id> <static-shp>?
<static-shp>      ::= '[' <int> (',' <int>)* ']'
<record-type>     ::= 'struct' <id> <shp>?
<record-field>    ::= <expr> '.' <id>
<full-constr>     ::= <id> '{' <expr> (',' <expr>)* '}'
<default-constr>  ::= <id> '{' '}'
<explicit-constr> ::= <id> '{' <explicit-set> (',' <explicit-set>)* '}'
<explicit-set>    ::= '.' <id> '=' <expr>
\end{verbatim}
\caption{EBNF for \sac{} record syntax.}
\label{fig:syntax}
\end{figure}

Record type declarations \verb|<record-decl>| start with the \texttt{struct} keyword, followed by the name of the record type.
Such a record can contain any positive number of fields, where each field can be of arbitrary type as long as the shapes of these field-types are statically fixed.
This includes record types as well, provided that there is no recursive use.

Record types \verb|<record-type>| extend the set of basic types\footnote{%
\texttt{byte},
\texttt{short},
\texttt{int},
\texttt{long},
\texttt{longlong},
\texttt{ubyte},
\texttt{ushort},
\texttt{uint},
\texttt{ulong},
\texttt{ulonglong},
\texttt{bool},
\texttt{char},
\texttt{float},
\texttt{double}
} in \sac{}. They can occur in all positions where basic types are admissible such as function signatures, variable declarations, or type assertions.
Similar to record type declarations, record types start with the \texttt{struct} keyword followed by the name of the record type.
Note here that the restriction to statically fixed shapes only applies to record fields, not the construction of record types.
A vector of statically unknown length of records \texttt{struct T} can be denoted by the type \texttt{struct T[.]}.

Fields of records \verb|<record-field>| can be accessed by using a dot-symbol in infix notation. Similar to the modification of array elements, we support the use of such infix-dot-symbols on the left hand side of assignments as well.

We provide three ways for constructing record values as reflected by the last non-terminals in the syntax extension
all of which constitute expressions in \sac{}.
First, \verb|<full-constr>| constructs a record value by enumerating a value for every field of the record, in the same order as those fields are defined in the corresponding record type declaration.
Second, \verb|<default-constr>| constructs a record value using default values for each of the fields. 
Examples of default values are $0$ for numbers and \textit{false} for booleans. 
The default value for an array is an array of the given shape, filled with the corresponding default values.
Third, \verb|<explicit-constr>| assigns values to selected fields and assigns the default value to the non mentioned fields. 
