
\section{Runtime Performance}\label{sec:case}

We investigate whether the record flattening transformation introduces any significant overhead compared to hand-optimised code by benchmarking two implementations of the n-body algorithm: a hand-optimised version without records, and a version that makes use of records.
In order to make optimal use of vectorization, the hand-optimised version operates on seven distinct arrays; three arrays for the x, y, and z coordinates, three arrays for the x, y, and z velocities, and a single array for the masses.

The record implementation from the example already does this through the use of the \texttt{Vector3} record, which highlights how without much effort we can play around with the memory layout.
We do however modify the velocity field to also use a \texttt{Vector3} instead of a \texttt{double[3]}.
The only necessary change then is to replace occurrences of \texttt{double[3]} types for velocities by \texttt{Vector3} types.

We benchmark these two implementations on 50,000 bodies, applying the timestep iteration 10 times.
These benchmarks are repeated ten times on a Dell PowerEdge R7525 rack server containing two AMD EPYC 7313 CPUs, which have 16 cores running at 3.7GHz.
Figure~\ref{fig:benchmark} plots the results of these benchmarks, which shows that the record version with \texttt{Vector3}s performs similarly to the hand-optimised version.

\begin{figure}[ht!]
    \centering
    \begin{tikzpicture}
        \begin{semilogxaxis}[
            xlabel=Threads,
            ylabel=Gflops/s,
            legend pos=north west,
            xticklabel style={/pgf/number format/fixed},
            yticklabel style={/pgf/number format/fixed},
            xtick={1,2,4,8,16,32},
            xticklabels={1,2,4,8,16,32},
        ]
            \addplot+[
                error bars/.cd,
                y explicit,
                y dir=both,
            ] table [
                x=p,
                y=mean,
                y error plus expr=\thisrow{stddev},
                y error minus expr=\thisrow{stddev},
            ] {data/orig.csv};
            \addplot+[
                error bars/.cd,
                y explicit,
                y dir=both,
            ] table [
                x=p,
                y=mean,
                y error plus expr=\thisrow{stddev},
                y error minus expr=\thisrow{stddev},
            ] {data/vec.csv};
            \legend{Hand-optimised,Vector3}
        \end{semilogxaxis}
    \end{tikzpicture}
    \caption{Compute rate in GFLOP/s.}
    \label{fig:benchmark}
\end{figure}
