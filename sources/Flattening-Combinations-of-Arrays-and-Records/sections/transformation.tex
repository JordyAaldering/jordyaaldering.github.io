
\section{Transformation}\label{sec:transform}

Similarly to earlier approaches (Homann et al.~\cite{SoA}, Jubertie et al.~\cite{AoSoA}, Kofler et al.~\cite{AoSoA2}) we aim to improve the runtime performance of programs with records by transforming arrays of records into records of arrays.
However, we go one step further and remove records from programs entirely.
As discussed in Section~\ref{sec:key-idea} this transformation decreases the number of allocations and reference counting operations, as well as improving memory locality.
An additional benefit is that removing records from programs in their entirety decreases the implementation effort of adding records to the language, since no modifications to the type system and other existing compiler phases are necessary.
Especially in a large-scale project such as \sac{}, with many compiler phases that would require modifications to support records, such a transformation is paramount for a feasible implementation effort.

After the record transformation has been applied to a program, that program will no longer contain any record types.
Instead, all record arguments and variable declarations are replaced by distinct arrays.
Record constructors and field accessors and mutators are transformed into functions that operate on arrays instead.
Any user-defined functions and primitive operations on records are similarly transformed into operations on arrays.

In the case of \sac{} this record transformation is actually split up into two separate phases.
During the parsing phase we replace records by a temporary ``external'' type, which is only fully removed after the type checking phase.
This ensures that error messages generated by the type checker still pertain to record types without having to add support for records to the type checker, and that these error messages remain consistent with the pre-existing error messages.
For example we get the following error message if we try to add a body to a string.
%
\begin{lstlisting}[language=red]
No definition found for a function "ArrayArith::+" that accepts
an argument of type "_MAIN::_struct_Body" as parameter no 1.
Full argument types are "( _MAIN::_struct_Body, String::string)".
\end{lstlisting}
%
However had records already been transformed into distinct arrays, the error message would look as follows and the relation between the error and the actual written code would be lost.
%
\begin{lstlisting}[language=red]
No definition found for a function "Array::+" that
expects 4 argument(s) and yields 1 return value(s)
\end{lstlisting}
%
For the sake of brevity however, we omit this additional step in the following transformation rules because it is specific to \sac{} and is not relevant to the actual transformation of records to arrays.

% ----------------------------------------------------------------------


\subsection{Constructors, Accessors, and Mutators}

The first step in transforming programs with records into programs without records is to replace record constructors, accessors, and mutators by generated function definitions.
For each record type in a program we generate a full constructor and a default constructor function, as well as accessor and mutator functions for every field of that record.
Any record constructors, accessors, and mutators in the program are then replaced by applications of the corresponding generated functions.

\subsubsection{Accessors and Mutators}

Normally an infix dot symbol is used to access or mutate the field of a record type, e.g. \texttt{bodies.pos}.
In order to access or mutate a nested field, multiple accessors or mutators may be chained.
However after programs have been transformed there will no longer be any records, and such a selection is no longer applicable.
Instead we generate accessor and mutator functions for every field occurring in a record type.
To access and mutate the position field of the body record, for example, these would be \verb|body_set_pos| and \verb|body_get_pos| respectively.
For a record type with $n$ fields, we generate an accessor and a mutator function for every field $i\in[1,n]$:
%
\begin{lstlisting}[escapechar=$]
type$\subb{1}$[shp$\subb{1}$], ..., type$\subb{n}$[shp$\subb{n}$]
rt_get_id$\sub{i}$(type$\subb{1}$[shp$\subb{1}$] id$\sub{1}$, ..., typen[shp$\subb{n}$] id$\sub{n}$)
{
    return id$\sub{i}$;
}

type$\subb{1}$[shp$\subb{1}$], ..., type$\subb{n}$[shp$\subb{n}$]
rt_set_id$\sub{i}$(type$\subb{i}$[shp$\subb{i}$] value, type$\subb{1}$[shp$\subb{1}$] id$\sub{1}$, ..., type$\subb{n}$[shp$\subb{n}$] id$\sub{n}$)
{
    return (id$\sub{1}$, ..., id$\sub{i-1}$, value, id$\sub{i+1}$, ..., id$\sub{n}$);
}
\end{lstlisting}
%
Note that the argument and return types in these generated functions can still be records at this point.
We expand these records at a later step, along with the expansion of records in user-defined functions.

Accessors and mutators through field selection can now be replaced by applications of these generated functions.
For a field selection \verb|id = x.y| on the right-hand-side of a let-expression we apply the accessor function \verb|id = rt_get_y(x)|, whereas if on the left-hand-side there occurs a field selection \verb|x.y = expr|, the mutator function is applied \verb|rt_set_y(expr, x)|.

\subsubsection{Constructors}

Syntactically there are three kinds of record constructors:
a full constructor that expects a value for each field in order,
a default constructor that takes no arguments and assigns a default (zero) value to each field,
and an explicit constructor with only the values for some fields explicitly defined.
For every record type \texttt{rt} in the program we generate a new function definition for both the full and the default constructor:
%
\begin{lstlisting}[escapechar=$]
type$\subb{1}$[shp$\subb{1}$], ..., type$\subb{n}$[shp$\subb{n}$]
new_rt(type$\subb{1}$[shp$\subb{1}$] id$\sub{1}$, ..., type$\subb{n}$[shp$\subb{n}$] id$\sub{n}$)
{
    return (id$\sub{1}$, ..., id$\sub{n}$);
}

type$\subb{1}$[shp$\subb{1}$], ..., type$\subb{n}$[shp$\subb{n}$]
zero(type$\subb{1}$[*] id$\sub{1}$, ..., type$\subb{n}$[*] id$\sub{n}$)
{
    return new_rt(genarray([shp$\sub{1}$], zero([:type$\subb{1}$])
                  ...
                  genarray([shp$\sub{n}$], zero([:type$\subb{n}$]));
}
\end{lstlisting}
%
Where \texttt{[:type]} is an empty array of the given type.
This ensures that the correct overload of the \texttt{zero} function is applied, returning the default value for that type.

Now whenever we encounter a full constructor of the form \texttt{rt\{expr\sub{1}, ..., expr\sub{n}\}} we replace it by \texttt{new\_rt(expr\sub{1}, ..., expr\sub{n})}, and whenever we encounter a default constructor of the form \texttt{rt\{\}} we replace it by \texttt{zero([:rt])}.
Finally, if we encounter an explicit constructor of the form
%
\begin{lstlisting}[escapechar=$]
rt{.field$\sub{q}$ = value$\sub{q}$, ..., .field$\sub{r}$ = value$\sub{r}$}
\end{lstlisting}
%
we first apply the default constructor, followed by a chain of mutator functions for the explicitly given fields.
%
\begin{lstlisting}[escapechar=$]
rt_set_id$\sub{r}$(value$\sub{r}$,
    ...
        rt_set_id$\sub{q}$(value$\sub{q}$, zero([:rt]));
\end{lstlisting}


\subsection{Expanding Records to Base Types}\label{sec:expand}

After record-specific syntax has been replaced by function applications, we must ensure that those and all other functions no longer contain any record types and record variables.
To achieve this we replace those record types and variables by the fields of those records instead.
We look at the \texttt{l2norm} function as an example:
%
\begin{lstlisting}
double
l2norm(struct Vector3 v)
{
    vx = vector3_get_x(v);
    vy = vector3_get_y(v);
    vz = vector3_get_z(v);
    return sqrt(vx * vx + vy * vy + vz * vz);
}
\end{lstlisting}
%
We aim to transform this function into one without any record types.
%
\begin{lstlisting}
double
l2norm(double x, double y, double z)
{
    vx = vector3_get_x(x, y, z);
    vy = vector3_get_y(x, y, z);
    vz = vector3_get_z(x, y, z);
    return sqrt(vx * vx + vy * vy + vz * vz);
}
\end{lstlisting}
%
In this example we see how the record argument \texttt{struct Vector3 v} is separated into three distinct arguments x, y, and z; corresponding to the three fields of the \texttt{Vector3} record.
Furthermore, all occurrences of this argument \texttt{v} are also replaced by the same three variables instead.

In the case where records and arrays are nested, this tranformation becomes non-trivial.
Consider the signature of the \texttt{timestep} function from the example.
Here the given record type is an array instead of a scalar value, furthermore it contains the nested \texttt{Vector3} record for the position, and an array \texttt{double[3]} for the velocity.
%
\begin{lstlisting}
struct Body[n]
timestep(struct Body[n] bodies, double dt)
\end{lstlisting}
%
After expanding the \texttt{Body} record, and the nested \texttt{Vector3} record, we expect the function signature to look like:
%
\begin{lstlisting}
double[n], double[n], double[n], /* pos  */
double[n,3],                     /* vel  */
double[n]                        /* mass */
timestep(double[n] x, double[n] y, double[n] z,
         double[n,3] vel, double[n] mass,
         double dt)
\end{lstlisting}
%
This example highlights multiple interesting cases.
Firstly, although the \texttt{mass} field of the body record is a scalar \texttt{double}, after expansion we expect it to become an array of type \texttt{double[n]} since \texttt{struct Body[n]} denotes an array of records instead of a scalar record, which consequently applies to all fields of the record.
Secondly, the body record contains a nested \texttt{Vector3} record, which itself is then expanded into its three distinct x, y, and z fields.
Similarly to the mass field, here we must also ensure that these fields become arrays of type \texttt{double[n]}.
Finally there is the velocity field, which itself is already an array of type \texttt{double[3]}.
In this case, the shape of the bodies array must be concatenated with the shape of the velocity field, resulting in the type \texttt{double[n,3]}.



\subsection{Denesting Fields of Nested Records}

In order to be able to apply this transformation we need to have a mapping of record types to the fully denested fields and expanded shapes.
We call this mapping $\sigma$, and allow it to be indexed by a record type to get the expanded fields of that record.
In the case of the \texttt{Vector3} record, this would look as follows:
%
\begin{align*}
    \sigma[\texttt{Vector3}] = \makeenv{
        \envid{x} \texttt{double},\ 
        \envid{y} \texttt{double},\ 
        \envid{z} \texttt{double}
    }
\end{align*}
%
Whereas for the \texttt{Body[n]} array of records we expect the following:
%
\begin{align*}
    \sigma[\texttt{Body[n]}] = \makeenv{
        &\envid{x} \texttt{double[n]},\ 
        \envid{y} \texttt{double[n]},\ 
        \envid{z} \texttt{double[n]} \\
        &\envid{vel} \texttt{double[n,3]},\ 
        \envid{mass} \texttt{double[n]}
    }
\end{align*}
%
Note that this environment does not contain the \texttt{pos} field, and instead has already expanded that record into its nested x, y, and z fields.

% ----------------------------------------------------------------------

\subsubsection{Denesting Records}

We populate this environment using a function called \texttt{Denest\textsubscript{rt}}.
This function expects a record declaration as its first argument, and the (initially empty) accumulated environment $\sigma$ as its second argument.
All fields of this record are then denested separately using a function \texttt{Denest\textsubscript{f}}, whose resulting environments are combined to form the mapping $\sigma$ of the record type.
We apply this denesting function to all record types in a program, from top to bottom.
\[
    \texttt{Denest\textsubscript{rt}}\left(\vcenter{\ttfamily\setlength{\baselineskip}{1em}
        \hbox{{\color{blue}struct rt} \{}
        \hbox{\quad {\color{blue}type\textsubscript{1}[shp\textsubscript{1}]}: id\textsubscript{1};}
        \hbox{\quad \dots}
        \hbox{\quad {\color{blue}type\textsubscript{n}[shp\textsubscript{n}]}: id\textsubscript{n};}
        \hbox{\};}
    },\ \sigma\right)
    = \vcenter{
        \hbox{\quad \texttt{Denest\textsubscript{f}}(%
            \texttt{\color{blue}type\textsubscript{1}},\ 
            \texttt{\color{blue}shp\textsubscript{1}},\ 
            \texttt{id\textsubscript{1}},\ 
            \ensuremath{\sigma}%
        )}
        \hbox{\ensuremath{\cup}\, \dots}
        \hbox{\ensuremath{\cup} \texttt{Denest\textsubscript{f}}(%
            \texttt{\color{blue}type\textsubscript{n}},\ 
            \texttt{\color{blue}shp\textsubscript{n}},\ 
            \texttt{id\textsubscript{n}},\ 
            \ensuremath{\sigma}%
        )}
    }
\]
%
Because from this point on this record may be used in a nested fashion in all following record declarations, we add this record to the environment.

% ----------------------------------------------------------------------

\subsubsection{Denesting Fields}

The main body of work lies in the \texttt{Denest\textsubscript{f}} function.
Given a field name $id$ and its corresponding type and shape, this function computes the environment $\sigma'$ of that shape.
Additionally this function requires the thus far accumulated environment $\sigma$, which is required when looking up the environment of a previously defined record type in the case that $type$ is a record type.
We distinguish between base-type fields and record type fields.
In the case that we encounter a base-type field $id$, be it a scalar or an array, a mapping can immediately be added to the environment without additional work.
%
\begin{align*}
    \texttt{Denest\textsubscript{f}}(basetype,\ [\,],\ id,\ \sigma)
        &= \{\, id:\,basetype \,\} \\
    \texttt{Denest\textsubscript{f}}(basetype,\ shp,\ id,\ \sigma)
        &= \{\, id:\,basetype[shp] \,\}
\end{align*}
%
If instead $id$ is a scalar record type, we lookup the previously computed environment of that record type.
Since this field does not have a shape, there is nothing more to do and we can copy the environment as is.
Because we require that records are defined top-to-bottom, this environment must exist at this point.
Otherwise an incorrect program was provided and we can raise an error.
%
\[
    \texttt{Denest\textsubscript{f}}(recordtype,\ [\,],\ id,\ \sigma)
        = \sigma[recordtype]
\]
%
As we have seen in the example, we need to do some additional work in the case that $id$ is an array of a record type.
Not only does that record type need to be denested, but the shape of $id$ must be prepended to all fields of the denested record type as well, for which we use a new function: \texttt{prepend}.
%
\[
    \texttt{Denest\textsubscript{f}}(recordtype,\ shp,\ id,\ \sigma)
        = \texttt{prepend}(shp,\ \sigma[recordtype])
\]
%
Following is the \texttt{prepend} function.
Its first argument is the shape we want to prepend, and the second argument is the environment of the record type to which we want to prepend this shape.
For every field in this environment, we then prepend the given shape to the previous shape, resulting in a new environment that has the same identifiers and types as the given argument, but now with expanded shapes.
%
\begin{gather*}
    \texttt{prepend}(shp_{rt},\ \sigma_{rt})
    = \left\{\, \vcenter{
        \hbox{\ensuremath{id_1:\,type_1[shp_{rt} :: shp_1],}}
        \hbox{\ensuremath{\dots,}}
        \hbox{\ensuremath{id_n:\,type_n[shp_{rt} :: shp_n]}}
    } \,\right\}\\
    \text{where}\ \sigma_{rt} = \{\, id_1:\,type_1[shp_1],\ \dots,\ id_n:\,type_n[shp_n] \,\}
\end{gather*}
%
No case distinction is needed for scalar fields, since their shape is the empty list (as seen in Section~\ref{sec:sac}), and thus the concatenation $shp :: []$ will act as an identity on $shp$.
Additionally we do not need to worry about nested records at this point, as they have already been denested and thus at this point we only have base-types.

Using this environment we can now actually apply the transformation proposed in Section~\ref{sec:expand}.
Whenever we encounter a record type, a record argument, or a record identifier we replace it by the expanded base-types and identifiers accordingly.


% ----------------------------------------------------------------------

\subsection{Primitive Functions}

This record argument expansion applies to both the formal arguments of a function definition, and the actual arguments of applications of those functions.
Consequently, the number of formal arguments and the number of actual arguments of user-defined functions will remain equal after the record transformation.
Because of this no additional work is required for user-defined functions.

However in the case of primitive functions this leads to a problem.
The actual record arguments of primitive function applications will have been expanded into multiple arguments, but since these functions are defined as compiler primitives they do not have a corresponding function definition in the program.
As a result, the number of actual arguments and the number of expected arguments for these primitive functions will no longer be the same.
Since we expose record types to users as primitive types, we should also ensure that primitive operations on records are also possible.
Namely, we must ensure that it is possible to get the dimensionality and shape of a record array, and it should be possible to select into this array.
Additional work is required with regards to primitive function applications in order to ensure that they remain valid after the record transformation.

Consider the built-in \texttt{shape} primitive that we use in the running example to find the upper bound of the tensor comprehension.
Given a single array, this primitive function returns the shape of that array.
Such primitive functions should be applicable to arrays of records as well, for example to get the shape of an array of bodies.
After the transformation, records arguments will have been expanded into multiple arguments, leading to incorrect applications of these primitive functions.
For example,
%
\begin{lstlisting}
shp = shape(bodies);
\end{lstlisting}
%
will be transformed into an application without records
%
\begin{lstlisting}
shp = shape(bodies_pos, bodies_vel, bodies_mass);
\end{lstlisting}
%
This transformed code is no longer valid.
The shape primitive expects only a single argument, however it now receives three arguments.
To resolve this we might decide to arbitrarily take the first field of the record, in this case \texttt{bodies\_pos}, and use only that value in primitive functions instead.
However, this field might already have a shape within the record itself.
Such is the case with \texttt{bodies\_pos}, where \texttt{pos} itself is already a three-element integer vector.
After transformation, this shape will then be \texttt{[n,3]}, whereas given the definition of \texttt{bodies} in the argument \texttt{struct Body[n] bodies}, we would expect its shape to be \texttt{[n]}.

Here we can rely on the fact that record fields are always arrays of a statically known shape.
Because we know that \texttt{pos} is a one-dimensional vector, we can statically decide that the last element of the resulting shape vector (\texttt{[3]}) should be dropped from the resulting shape.
%
\begin{lstlisting}
shp = drop(-1, shape(bodies_pos));
\end{lstlisting}
%
We apply a similar approach to the remaining primitive functions that require modification, such as \texttt{dim} (dimensionality) and \texttt{sel} (selection).
However for the sake of brevity we omit those cases here.

% ----------------------------------------------------------------------

\subsection{Tensor Comprehension}

Whereas tensor comprehensions on records previously operated on only that single record value, after the record expansion these tensor comprehensions operate on and return multiple values.
For example, a tensor comprehension that generates a list of bodies
%
\begin{lstlisting}
bodies = { iv -> Body{} | iv < [N] };
\end{lstlisting}
%
is transformed into a tensor comprehension that returns three values:
%
\begin{lstlisting}
pos, vel, mass = { iv -> ([0,0,0], [0,0,0], 0) | iv < [N] };
\end{lstlisting}
%
This requires that tensor comprehensions, and similar constructions such as with-loops, are able to operate on and return multiple values.
In the case of \sac{}, this is already supported~\cite{sac-scan}.
