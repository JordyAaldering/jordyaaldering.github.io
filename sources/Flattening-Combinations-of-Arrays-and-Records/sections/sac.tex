
\section{Single assignment C}\label{sec:sac}

Since we use the functional array language Single~assignment~C (\sac{})~\cite{sac,sac2} as implementation vehicle for our proposed flattening,
we provide a quick overview of the key features that we use throughout the paper.
For a more in-depth introduction we refer the reader to papers such as~\cite{sac,sac2} or the online material available on \url{https://sac-home.org/docs:main}.

\sac{} combines a syntax very close to the imperative language C with a purely
side-effect-free semantics. This is achieved by the addition of multi-dimensional, immutable arrays to the language core, paired with an exclusion of pointers.

\subsection{Arrays in \sac{}}

Arrays in \sac{} are multi-dimensional; each array has a dimensionality (also referred to as \textit{rank}) and a \textit{shape} which describes the legal index-space of the array.
This enables computations on multi-dimensional arrays to inspect and 
manipulate not only the data of an array but also its rank and shape.
Table~\ref{tab:array} shows a few examples for arrays in \sac{}. The left-most column provides
an abstract / mathematical representation, the middle column contains a corresponding definition in \sac{} and in the right column we see the three conceptual components of these arrays: their rank (an integer value), their shape (a vector of natural numbers), and their data (a vector of the values of the array). 

\begin{table}[ht!]
    \newcommand{\repr}[3]{%
        \begin{tabular}{ll}
            rank:  & #1   \\
            shape: & [#2] \\
            data:  & [#3]
        \end{tabular}\vspace{2pt}\\\hline\noalign{\vskip 2pt}}
    \centering
    \begin{tabular}{c@{\hskip 1em}c@{\hskip 1em}l}\hline\noalign{\vskip 2pt}
        $\begin{pmatrix}
            1 & 2 & 3 \\
            4 & 5 & 6 \\
            7 & 8 & 9
        \end{pmatrix}$
            & \texttt{[[1,2,3], [4,5,6], [7,8,9]]}
            & \repr{2}{3,3}{1,2,3,4,5,6,7,8,9}
        $\begin{pmatrix}
            1, & 2, & 3, & -4, & -5
        \end{pmatrix}$
            & \texttt{[1,2,3,-4,-5]}
            & \repr{1}{5}{1,2,3,-4,-5}
        $\begin{pmatrix}\begin{pmatrix}
            true, & false, & true
        \end{pmatrix}\end{pmatrix}$
            & \texttt{[[true, false, true]]}
            & \repr{2}{1,3}{true,false,true}
        0.5 
            & \texttt{0.5}
            & \repr{0}{\,}{0.5}
    \end{tabular}
    \caption{Array representation in \sac{}}
    \label{tab:array}
\end{table}

\noindent
From this table, we can observe that all elements of any given array ultimately can be represented as a \textit{flat} data vector in memory.
While the \sac{} syntax in the middle column may suggest to the programmer that some arrays indeed are nested, in fact, they are not; in \sac{}, nesting inherently is indistinguishable from higher-dimensional arrays.
This contrasts starkly with the idea of arrays prevalent in most non-array languages, where arrays are considered inherently one dimensional and where higher dimensional arrays are mimicked through nesting.
As a consequence of \sac{}'s choice to support higher-dimensional arrays in combination with a vector for representing the shape, nested notations in \sac{} are restricted to the homogeneous case, i.e. all inner expressions must have the same shape.
An expression of the form \texttt{[[1],\,[2,3]]} is not admissible in \sac{}.

\subsection{Tensor Comprehensions}

The key language construct of \sac{} for defining arrays from other arrays is the \textit{Tensor Comprehension}~\cite{sac-tensor,sac-scan}.
A tensor comprehension is essentially a mapping of index vectors to values.
For example, the tensor comprehension
%
\begin{lstlisting}
{ iv -> arr[iv] * 2  |  iv < shape(arr) }
\end{lstlisting}
%
defines an array of the same shape as array \texttt{arr}, whose value in each index position \texttt{iv} equates to twice the value of \texttt{arr} at the given position.
Note here that the index variable \texttt{iv} denotes an index vector with as many elements as the array \texttt{arr} has dimensions.
The ability to express such operations that operate on arrays of arbitrary rank is referred to as \textit{rank polymorphism}.
If a fixed rank is expected, an explicit vector of indices can be used instead.
For example,
%
\begin{lstlisting}
{ [i,j] -> mat[j,i]  |  [i,j] < reverse(shape(mat)) }
\end{lstlisting}
%
transposes the two dimensional array \texttt{mat}.
Despite having a fixed-rank index space, this tensor comprehension can be turned into a rank polymorphic operation by a small change into
%
\begin{lstlisting}
{ [i,j] -> mat[j,i]  |  [i,j] < reverse(take([2], shape(mat))) }
\end{lstlisting}
%
The rank polymorphism of this expression is achieved through two measures.
Firstly, selections in \sac{} select entire subarrays if there are fewer indices than the rank of the array to be selected from.
Secondly, the specification of the upper bound through \texttt{reverse(take([2], shape(mat)))} ensures a two element vector as upper bound, even if the rank of the array \texttt{mat} is higher than 2.

\subsection{Function Definitions, Types, and Type Patterns}

Function definition in \sac{} look like their C counterparts.
The main difference is the types that are supported in \sac{}.
All types in \sac{} consist of an element type followed by a shape specification.
If the shape specification is omitted, scalar arrays are expected.
For example, the simple transpose can be defined as:
%
\begin{lstlisting}
double[n,m] transpose(double[m,n] mat)
{
    return { [i,j] -> mat[j,i]  |  [i,j] < reverse(shape(mat)) };
}
\end{lstlisting}
%
Note here that the shape variables \texttt{m} and \texttt{n} define scalar integer values that can be referred to in the function body.
This allows for a specification of the upper bound as \texttt{[n,m]} instead of \texttt{reverse(shape(mat))}.
%
\begin{lstlisting}
double[n,m] transpose(double[m,n] mat)
{
    return { [i,j] -> mat[j,i]  |  [i,j] < [n,m] };
}
\end{lstlisting}
%
The dual use of shape variables for describing domain constraints as well as for capturing these values are referred to as \textit{type patterns}~\cite{sac-typepattern}.
