
\section{Related Work}

\paragraph{Dynamic adaptation algorithms.}
Besides the runtime-based dynamic adaptation algorithm\cite{sac-mtdynamic}, there are more
approaches that steer the resource allocation of a system through dynamic control, in an attempt to
improve runtime performance.

Grand Central Dispatch (GCD) is a technology developed by Apple that aids developers in writing
parallel
programs~\cite{grand-central-dispatch-queues,grand-central-dispatch2,grand-central-dispatch}. GCD is
integrated into the host operating system and provides an integrated approach for thread management,
shifting this load from the developer to the scheduler. However, GCD still requires some manual
instrumentation from developers. They must define blocks of code that are dispatched to GCD
synchronously or asynchronously, using one of several types of waiting queues. Parallel Dispatch
Queues (PDQ) are a similar programming abstraction~\cite{dispatch-queues}. By synchronizing messages
in a queue, it aims to reduce the overhead caused by acquiring and releasing synchronization
primitives, as well as preventing busy waiting within handlers. These approaches do not consider
energy-efficiency, and only optimize for runtime performance.

Related is Dynamic Voltage Frequency Scaling (DVFS). Instead of trying to increase energy-efficiency
or runtime performance by instrumenting the software, this approach aims to instrument the hardware
by scaling the voltage supplied to the CPU. Dynamic programming approaches have been applied to
further improve the effectiveness of DVFS~\cite{dvfs-dynamic-programming}.

\paragraph{Profile-based optimization.}
Profile-based optimizations use the results of profiling runs to select an optimal implementation of
an algorithm for the given input or hardware configuration. A prominent example of this is FFTW3,
which selects one of many discrete Fourier transformation implementations through
profiling~\cite{fftw3}. They achieve this through the use of a special-purpose compiler, that
generates optimized code for the given hardware in a planning phase. It would be interesting to
investigate whether similar results can be obtained by optimizing for energy-efficiency instead.

Profile-based optimization of virtual machine scheduling is a popular avenue of research in the
context of data-centres~\cite{profile-based1,profile-based2,profile-based3,profile-based4}. These
approaches focus on the development of resource allocation policies and scheduling algorithms that
aim to decrease the carbon footprint of data-centres without compromising on the required quality of
service.

\paragraph{Static methods for optimizing energy.}
Although there exists a myriad of compiler optimizations that aim to minimize the runtime of
programs, not many optimizations exist that aim to specifically reduce energy
consumption~\cite{compiler-energy-differences}. Pallister et~al. have provided several optimizations
that aim to reduce the energy consumption of programs, in the context of embedded
devices~\cite{compiler-energy-flash,compiler-energy-methods}. As have we, they have found that for
optimal energy-efficiency gains a vertical integration process that exploits hardware-specific
energy characteristics is needed, to bridge the gap between hardware and software.

Bangash et~al. have investigated byte-code transformations in the context of Android applications,
determining whether certain (combinations of) transformations lead to in increase or decrease in
energy consumption~\cite{compiler-energy-android}. They found that certain combinations of byte-code
transformations can actually increase energy consumption, whereas choosing the right combination of
transformation can lead to a reduction of up to $11\%$ in energy consumption.

\paragraph{Energy visualization \& analysis.}
One of the key requirements in allowing developers to increase the energy-efficiency of their
software is the ability to visualize and analyse the energy consumption of their
code~\cite{van2023organizational}. There exist semantics for programming languages that can
calculate the energy consumption of their programs, and allow for calculation of break even points
of algorithms~\cite{symbolic-dependent-types,symbolic-dependent-types2,phdBernard}. However, making
these semantics work for a full language is hard~\cite{energy-analysis}.

Another approach uses model checking to detecting energy bugs and hotspots in control
software~\cite{symbolic-energy-bugs}. The control software targetted with this model does have key
interaction between software and hardware made explicit in the source code This is not the case for
our target domain, in which all interactions are implicit. This makes using model checking harder.
Using the same explicit interaction, energy consumption graphs can be generated next to control
software source code inside an IDE~\cite{symbolic-skylines}. Although it allows for near instant
feedback to programmers, which is great for improving the code, it does not fit our target domain.

\paragraph{Energy-optimizing strategies.}
There is a handful of programming patterns that are known to improve the energy-efficiency of
programs. These include applica\-tion-level techniques such as caching, buffering, and batching, but
also system-wide techniques such as retention policies and data compression. The effectiveness and
adoption of such techniques have been previously
investigated~\cite{strategies-hpc,strategies-industry}.

Another prominent example of a well-known pattern for reducing energy consumption is load balancing.
Multiple approaches exist, however these make generalized assumptions about the energy-efficiency of
the hardware. Kistowski et~al. have compared a variety of load distribution strategies, and propose
a new strategy that reduces a system's energy consumption even
further~\cite{strategies-load-distribution}.
