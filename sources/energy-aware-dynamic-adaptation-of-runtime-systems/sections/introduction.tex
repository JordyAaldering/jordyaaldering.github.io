
\section{Introduction}

The importance of software sustainability has grown rapidly in recent years, alongside an increasing awareness of environmental concerns.
From mobile applications to data centres, minimizing the energy consumption of software has become essential for mitigating environmental impact, increasing battery life, and reducing operational costs.
Traditional software systems often focus on maximizing runtime performance without adequately considering energy consumption.
However, as sustainability becomes a priority, there is a growing need for approaches that balance runtime performance and energy-efficiency.

The current landscape of computing hardware is characterized by its heterogeneity, with a shift towards multicore and many-core systems.
High-perfor\-mance computing applications typically aim to fully utilize all available resources on these systems, with the goal of maximizing runtime performance.
Determining efficient resource allocation for optimizing performance -- be it runtime or energy-efficiency -- depends not only on an algorithm's implementation, but also on input data size, hardware characteristics, cache utilization, and system-specific configurations such as thread pinning.
This makes static approaches for determining resource allocation increasingly difficult, as they fail to account for variations in hardware capabilities~\cite{heterogeneous-systems}.
Furthermore, it has been shown that optimizing for runtime is not always equivalent to optimizing for energy-efficiency, and that optimizing runtime can have a negative effect on energy consumption~\cite{compiler-energy-android,compiler-energy-differences}.

A key candidate for improving the energy-efficiency through more efficient resource allocation is the thread management of an application.
Traditionally, thread management in software systems has focused primarily on performance metrics such as execution time and operations per second.
A notable approach in the context of the Single assignment C (\sac{}) language uses a dynamic adaptation algorithm that adjusts the thread-count of a program by observing changes in runtime metrics~\cite{sac-mtdynamic}.
This technique enables \sac{} applications to adapt to varying workloads and resource availability, improving runtime performance by efficiently utilizing computational resources.
This method optimizes for runtime performance, but does not directly address energy consumption.

We present a method that extends existing runtime adaptation techniques by incorporating energy consumption as the primary factor in decision-making.
By dynamically adjusting the number of threads based on real-time energy consumption metrics, we show that it is possible to achieve a more sustainable balance between runtime performance and energy-efficiency.
Functional languages are an especially suitable target for this dynamic adaptation algorithm, as their side-effect-free nature and strong semantic guarantees allow for easier code generation and straightforward scalability of the parallelism of programs.

Our contributions are:
\begin{itemize}
    \item A static analysis of data-parallel programs that investigates the relation between the level of parallelism and overall energy consumption.
    (Section~\ref{sec:observations})
    \item A dynamic adaptation algorithm that aims to minimize energy consumption of data-parallel programs at runtime through external control.
    (Section~\ref{sec:implementation})
    \item An implementation of a thread-steering mechanism in the data-parallel functional language \sac{}, based on this adaptation algorithm.
    (Section~\ref{sec:implementation-sac})
    \item An evaluation of how close the adaptation algorithm comes to the energy-efficiency of a theoretical optimum.
    (Section~\ref{sec:evalation-adaptation} and \ref{sec:evalation-oracle})
    \item An evaluation of the energy-efficiency of the adaptation algorithm compared a runtime-based approach and static approaches.
    (Section~\ref{sec:evalation-runtime} and \ref{sec:evalation-static})
\end{itemize}
