
\section{Conclusion}

We present a dynamic adaptation algorithm aimed at minimizing the energy consumption of data-parallel applications.
It is based on the observation that data-parallel programs typically apply the same parallel operations in a repetitive fashion.
This property allows for a dynamic feedback loop that reacts to changes in energy consumption from one repetition to the next.
Our experiments show that such a non-intrusive method, in a functional context where scaling the level of parallelism is straightforward, can be very effective while requiring no in-depth program analysis, involving no programmer effort at all and causing negligible overhead.
Looking at four different use cases with dynamically changing behaviour, we can see that our adaptation approach in most cases gets within 15\% of a manually generated, ideally adapting runtime.
The only remarkable outlier is the case of matrix multiplication where the ideal case depends on the thread pinning strategy, which our adaptation algorithm fails to find.
However, we also find that optimizing these cases is highly context-dependent and that slight variations in configuration or implementation can introduce significant changes to the energy consumption characteristics of a program.

Furthermore, we observe that the dynamic approach yields energy improvements over static approaches.
Depending on the number of threads chosen, the gains are up to 79\% for any reasonable choice, and up to 100\% if a poor decision is made.
We also notice that in many cases our energy-optimising algorithm improves slightly over the runtime-optimising approach.
This is reflected by our static analyses, which show that, while there is a correlation between overall runtime and energy use, the energy minima usually are reached before runtime minima when increasing the number of threads.
Overall, we show that our method provides a feasible method for lowering the energy footprint of data-parallel applications solely based on automatic code instrumentation.

Whilst our adaptation algorithm proves effective across a range of data-parallel workloads, future work could explore the capability of the algorithm to adapt to changes in runtime conditions caused by other processes running in the background.
Furthermore, it would be interesting to investigate whether the same approach can be applied to switch between devices in a heterogeneous system, such as CPU and GPU implementation of an algorithm, or to switch between thread pinning strategies.
Beyond the array based domain, there is potential for applying this approach in other contexts that handle a large amount of similar computational complexity.
For example, by adjusting resources dynamically in response to workload demands, web-servers could decrease their energy consumption, potentially without compromising on responsiveness.
