
\section{Energy-Aware Dynamic Adaptation}\label{sec:implementation}

In Section~\ref{sec:observations} we determine a multitude of running conditions that affect the selection of a thread-count for maximal energy-efficiency.
Given the multitude of effects, a static cost-model-based approach seems infeasible.
A profiling-based approach might provide better data but would incur a significant energy overhead.
Consequently, we look for a solution that avoids the need for manual tuning by dynamically adapting to runtime changes.

The overall idea is based on the observation that data-parallel codes typically perform several repetitions of the same parallel computation.
This behaviour allows dynamic adaptation to tune the parallelism between such repetitions~\cite{sac-mtdynamic}.
Our adaptation algorithm monitors changes in the energy consumption from one repetition to the next, and periodically adjusts the recommended thread-count to find the thread-count with the lowest energy consumption.
Based on measured changes in energy consumption, using power meters built in to the processor, the algorithm determines whether to increase or decrease the thread count for the next iteration, and by what amount.
The hypothesis is that with this approach, we can adapt the thread-count during runtime in correspondence to changes in energy consumption, converging to a (local) optimum for energy-efficiency.

The algorithm recomputes the recommended thread-count at a fixed interval; waiting for a fixed amount of energy measurements to arrive.
Each of these energy measurements is the energy consumed during a single iteration of the parallel region, measured using RAPL (Section~\ref{sec:defining-energy}).
The frequency is configurable, but we find for our benchmarks that a frequency of ten iterations per thread-count adjustment strikes a balance between a high adaptation speed whilst also being resilient against noise such as short-running background tasks and operating system overhead.
The algorithm uses two variables for steering the thread-count of an application: a step direction, and a step size.
The step direction describes whether the number of threads will be increased or decreased, corresponding to a value of $1$ or $-1$.
The step size gives the amount of threads that will be added or removed from the current thread-count, respectively.

When the median energy consumption of an iteration differs more than a factor $\alpha$ compared to the energy consumption of the previous iteration, we hypothesize that there is likely a change in workload behaviour or system configuration.
To quickly converge to the new optimal thread-count, the step size is reset to half the number of maximum threads.
For our benchmarks we observe that a value of $0.5$ for $\alpha$ works well.
Otherwise, if there was an increase in energy consumption that is less than factor $\alpha$, this means that the previous iteration was more energy-efficient.
We predict that the optimal thread-count is somewhere in between the current thread-count and the previous thread-count.
Therefore, the step direction is inverted, and the current step size is halved.

As we observe in Section~\ref{sec:observations}, changing the thread-count by even a single thread can have a significant negative impact on the energy consumption of a program.
In an attempt to minimize this overhead, alongside the intrinsic scheduling overhead for making changes to the thread-count, we aim to settle into a fixed thread-count when we are close to the (local) optimum.
To this end we use a real-valued step size that nears zero as we get close to this optimum, as opposed to a positive integer step size that always results in a change to the thread-count.
Whereas typically the step size is halved in each iteration, as it reaches zero we gradually decrease it by a lower amount, ensuring that changes to the thread-count do still occur occasionally.
This change to the step size is defined by a function $\Delta t\, /\, (\Delta t + \beta)$, where a lower value for $\beta$ results in smaller changes to the step size.
For our benchmarks we find that a value of $0.85$ for $\beta$ works well.

Eventually the step size reaches zero, and the thread-count will stagnate.
In an attempt to escape potential local optima, and to adapt to subtle changes to the energy consumption pattern of a program, the step size is reset to half the maximum number of threads when it becomes less than $\gamma$.
The lower the value of $\gamma$, the more iterations are required until the step size is reset.
Given our choice for $\beta$, a value of $0.155$ for $\gamma$ works well for our benchmarks.

A formal definition of this algorithm is described in Algorithm~\ref{alg:controller}.

\begin{algorithm}[!ht]
    \begin{algorithmic}[1]
            \State $\hat{d} \gets -1$
            \State $\Delta t \gets m_t\, /\, 2$
            \State $E_{old} \gets 0$
            \Loop
                \State $S \gets$ Wait for samples
                \State $E_{new} \gets median(S)$
                \If{$E_{new}\, /\, E_{old} < (1 - \alpha)\, \vee\, E_{new}\, /\, E_{old} > (1 + \alpha)$}
                    \State $\hat{d} \gets sign(m_t - 2 n_t)$
                    \State $\Delta t \gets m_t\, /\, 2$
                \Else
                    \If{$E_{new} > E_{old}$}
                        \State $\hat{d} \gets -\hat{d}$
                    \EndIf
                    \If{$\Delta t > \gamma$}
                        \State $\Delta t \gets max(\Delta t\, /\, 2,\, \Delta t\, /\, (\Delta t + \beta))$
                    \Else
                        \State $\hat{d} \gets sign(m_t - 2 n_t)$
                        \State $\Delta t \gets m_t\, /\, 2$
                    \EndIf
                \EndIf
                \State $E_{old} \gets E_{new}$
                \State $n_t \gets n_t + \hat{d} \cdot \Delta t$
                \State $n_t \gets max(1,\, min(m_t,\, n_t))$
            \EndLoop
    \end{algorithmic}%
    \caption{Algorithm for repeatedly updating the thread-count $n_t$ based on energy measurements $S$, using step direction $\hat{d}$ and step size $\Delta t$.}
    \label{alg:controller}
\end{algorithm}

A drawback of this approach is that it contains manually tuned variables, which might need reconfiguration in different contexts.
The main difficulty lies in the variation in energy-efficiency between hardware systems, and in run-to-run variance due to input-dependent behaviour and changes to the software environment.
In future work we aim to investigate whether, within this context, a control theory-driven design technique can help mitigate this drawback~\cite{control-theory}.
