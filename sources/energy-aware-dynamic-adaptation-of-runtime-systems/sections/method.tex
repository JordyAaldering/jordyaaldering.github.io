
\section{Defining ``energy consumption''}\label{sec:defining-energy}

To quantify the energy-efficiency of a program, it is essential that we accurately define what we
mean by its ``energy consumption''. Ideally we would isolate the total amount of energy consumed by
the hardware to run one specific program, separating it from other sources of energy consumption
such as operating system overhead, background tasks, and the baseline power required to keep the
hardware operational. However, fully isolating the energy consumption of a single process remains a
challenge.

Currently, separating a program's energy consumption from that of other programs is only possible in
certain environments. On Apple Silicon, for example, this is possible through the use of the
\verb|task_info| API, which returns a per-process energy consumption value~\cite{firefox-profiling}.
For Apple's Intel-based machines, the \verb|diagCall64| function can be used instead. On Linux-based
operating systems, separating the energy consumption is theoretically possible by extending the
scheduler statistics, however to our knowledge there is no implementation available. Although these
methods provide potential solutions, they lack necessary documentation and would require specific
hardware and elevated permissions, which go against our goal of making a broadly applicable
adaptation algorithm. Furthermore, these methods still fail to exclude the baseline power required
to keep the hardware operational, for which currently no solution exists.

For the purposes of this study we define the energy consumption of a program as the total amount of
energy consumed by the CPU throughout the entire runtime of that program, given that no other
user-defined programs are running on that device. This implies that operating system overhead and
baseline power required to keep the hardware operational are included in these measurements, as we
cannot control or isolate these. We hypothesize that: if we succeed in improving the
energy-efficiency of a program, this will result in a measurable decrease in total energy
consumption across the runtime of that program.

\subsection{Measuring energy consumption}

Running Average Power Limit (RAPL) is a technology developed by Intel to measure and control the
energy consumption of the CPU. It can be used to measure the total amount of energy consumed by the
CPU throughout the runtime of a program. Although initially based on estimation models, thanks to
improvements in hardware support, RAPL has evolved into a precise energy measurement tool. RAPL
reports fine-grained and reliable energy consumption data across various CPU domains, such as the
package, PSys, and DRAM~\cite{rapl}. RAPL operates using a running counter that tracks the
cumulative energy consumed by each of these domains. The energy consumption in micro-Joules over a
time interval is calculated as the difference in counter values at the start and end of that time
period. RAPL is available on Intel CPUs since the Sandy Bridge architecture (2011), and AMD supports
a similar feature starting with their Zen architecture (2017). In addition to its wide availability
over the past decade, RAPL requires only slightly elevated privileges~\cite{rapl-security}, making
it accessible on most systems.
