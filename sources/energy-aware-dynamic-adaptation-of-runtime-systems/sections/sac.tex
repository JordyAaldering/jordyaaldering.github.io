
\section{Single assignment C}\label{sac}

Single assignment C (\sac{}) is a functional array language that resembles the imperative syntax of
languages such as C, whilst remaining side-effect-free~\cite{sac1,sac2}. It aims to generate
high-performance parallel code for a wide range of multicore and many-core architectures from a
single source code.

One of the key design choices that makes this possible is the use of a single language construct for
all array operations, named a tensor comprehension~\cite{sac-tensor}. These tensor comprehensions
are side-effect-free mappings from index spaces to element values, based on the set builder
notation.

Take for example the following tensor comprehension:
\begin{verbatim}
res = { iv -> arr[iv] + 1 | [0,0] <= iv < [9,9] };
\end{verbatim}

It constitutes an index vector \verb|iv| that ranges over the index set with lower bound
\verb|[0,0]| and upper bound \verb|[9,9]|. If the lower or upper bound can be inferred from the
usage of \verb|iv| in the inner expression, they may be left out. In the example, the upper bound
would be inferred as the shape of \verb|arr|. For each index vector in this range, the result is a
selection into the array \verb|arr| incremented by one.

All array types in \sac{} consist of an element type followed by a shape specification in square
brackets. In simple cases, such shape specifications are lists of numbers describing the length of
each dimension of the array. To express relations between shape components, these shape lengths can
be replaced by variables, where identical variable names require identical lengths in the
corresponding positions. For example, an integer vector of length $100$ can be defined as
\verb|int[100]|, and a square integer matrix of size $n \times n$ can be referred to as
\verb|int[n,n]|. A multi-dimensional array of doubles with rank $d$ and shape vector $shp$ can be
described by \verb|double[d:shp]|. This notation is referred to as type
patterns~\cite{sac-typepattern}. It allows not only to express constraints within shapes but also
between different function arguments and return types. Take for example the shape constraints of a
matrix multiplication. It requires that the multiplied matrices are of shapes $u \times v$ and $v
\times w$, resulting in a $u \times w$ matrix.

This can be expressed as:
\begin{verbatim}
double[u,w] matmul(double[u,v] a, double[v,w] b)
\end{verbatim}

Furthermore, shape variables such as \verb|u|, \verb|v|, and \verb|w| can be used in function bodies
as if they are user-defined variables. In combination with variable-rank type patterns, this allows
for the embeddings of complicated shape expressions within type patterns, simplifying the function
bodies.

\subsection{Parallelism in Single assignment C}\label{sac-par}

The \sac{} compiler is able to generate parallel code for a range of parallel architectures,
including multicore machines~\cite{sac-shared}, GPUs~\cite{sac-gpu,sac-gpu2}, and
clusters~\cite{sac-mpi}. In this paper, we extend the work on code generation for multicore
machines.

Before the actual generation of multithreaded code, the high-level optimization phase of the
compiler extensively fuses tensor comprehensions to increase the granularity of parallelism and to
improve the locality of code~\cite{sac1}. The multicore back-end then generates code to execute all
top-level tensor comprehensions of sufficiently large size in parallel~\cite{sac-shared}. Each of
these regions is executed in a fork-join style of bulk synchronous computing. To avoid the overhead
of thread creation and termination for each such parallel region, threads are created only once upon
program startup and kept active throughout the execution.

We refer to such a set of threads as \textit{bees} in a \textit{bee hive}. One designated thread,
the \textit{queen bee}, executes the entire program including the sequential sections, and
coordinates all the other bees. The other bees wait until the start of a parallel section is
signalled to them by the queen, and they return to the waiting state thereafter. More details on the
multithreaded back-end can be found in~\cite{sac-shared}.
