\documentclass[11pt,a4paper]{report}
\usepackage[utf8]{inputenc}
\usepackage[a4paper]{geometry}
\usepackage[nottoc,numbib]{tocbibind}

\usepackage[parfill]{parskip}
\setlength{\parindent}{0pt}

\usepackage{titlesec}
\titleformat{\chapter}{\Large\bfseries}{\Huge\thechapter\quad}{0pt}{\Huge}

\usepackage{bm}
\usepackage{url}
\usepackage{pbox}
\usepackage{amsmath}
\usepackage{latexsym}
\usepackage{syntax}
\usepackage{semantic}
\usepackage{minted}

\usepackage{tikz,tikz-qtree,pgfplots}
\usetikzlibrary{arrows,positioning,calc,automata,trees}
\pgfplotsset{compat=1.16}

\usepackage{graphicx}
\graphicspath{{images/}}

% ===========================================================================

% Grammar
\newcommand{\valpair}[2]{\langle #1,\, #2 \rangle}
\newcommand{\sempair}[2][\sigma]{\left( #2,\, #1 \right)}
\newcommand{\sempairb}[2][\sigma]{\bm{\left(} #2,\, #1 \bm{\right)}}
\newcommand{\DownA}{\makebox[14.9pt][c]{$\Downarrow$}}
\newcommand{\epssig}{\varepsilon.\sigma}

% Expressions
\newcommand{\lambdaexpr}[2]{\lambda #1.\, #2}
\newcommand{\applyexpr}[3]{(\lambda #1.\, #2)\; #3}
\newcommand{\letexpr}[3]{\text{let}\, #1\hspace{-1pt} =\hspace{-1pt} #2\, \text{in}\, #3}
\newcommand{\condexpr}[3]{\text{if}\; #1\, \text{then}\; #2\, \text{else}\; #3}
\newcommand{\withexpr}[6]{\pbox[c]{\textwidth}{\relax\ifvmode\centering\fi
    gen $#1$ $#2$ \\ with $#3\hspace{-1pt} \leq\hspace{-1pt} #4\hspace{-1pt} <\hspace{-1pt} #5$}\, \text{in}\, #6}
\newcommand{\genexpr}[2]{\text{gen}\, #1\; #2}
\newcommand{\bopexpr}[2]{#1\; \texttt{bop}\; #2}
\newcommand{\uopexpr}[1]{\texttt{uop}\; #1}
\newcommand{\selexpr}[2]{\text{sel}\; #1\, #2}
\newcommand{\shpexpr}[1]{\text{shape}\; #1}
\renewcommand{\dimexpr}[1]{\text{dim}\; #1}

% Inference & Rewrite
\newcommand{\F}{\mathcal{F}}
\renewcommand{\S}{\mathcal{S}}
\newcommand{\D}{\mathcal{D}}
\newcommand{\N}{\mathcal{N}}
\newcommand{\X}{\mathcal{X}}

\newcommand{\sd}[3][\gamma]{\mathcal{SD}\hspace{-1.2pt} \left( #2,\, #3,\, #1 \right) \hspace{-1pt}}
\newcommand{\pv}[2][\gamma]{\mathcal{PV}\hspace{-1.2pt} \left( #2,\, #1 \right) \hspace{-1pt}}
\newcommand{\reduce}[3][\varepsilon]{\mathcal{#2}\hspace{-1.2pt} \left( #3,\, #1 \right) \hspace{-1pt}}

% Helpers
\newcommand{\oplusnl}{\\&\hspace{1em}\oplus}
\newcommand{\dem}[2]{\text{#1}\colon\hspace{-2pt}#2}
\newcommand{\envmap}[3]{#1\{\text{#2}\hspace{-2pt}\to\hspace{-2pt}#3\}}

\makeatletter
\newcommand{\pushleft}[1]{\ifmeasuring@#1\else\omit$\displaystyle#1$\hfill\fi\ignorespaces}
\makeatother

\usepackage{subfiles}
\providecommand{\main}{.}

% ===========================================================================

\begin{document}
\begin{titlepage}
\begin{center}
\textsc{\LARGE Bachelor thesis \\ Computing Science} \\[1.5cm]
\includegraphics[height=100pt]{logo}

\vspace{0.4cm}
\textsc{\Large Radboud University} \\[1cm]
\hrule
\vspace{0.4cm}
\textbf{\huge A new semantics for array programming languages; how to introduce some laziness without being lazy} \\[0.8cm]
\hrule
%
\vspace{2cm}
\begin{minipage}[t]{0.45\textwidth}
\begin{flushleft} \large
\textit{Author:} \\
    J.M. (Jordy) Aaldering, %\\
    s1004292 \\
    \texttt{j.aaldering@student.ru.nl}
\end{flushleft}
\end{minipage}
%
\begin{minipage}[t]{0.45\textwidth}
\begin{flushright} \large
\textit{First supervisor/assessor:} \\
    Prof. dr. S.B. (Sven-Bodo) Scholz \\
    \texttt{svenbodo.scholz@ru.nl}\\[1.3cm]
\textit{Second assessor:} \\
    dr. P.W.M. (Pieter) Koopman \\
    \texttt{pieter@cs.ru.nl}
\end{flushright}
\end{minipage}
%
\vfill
{\large March 17, 2021}
\end{center}
\end{titlepage}

% ===========================================================================

\subfile{sections/abstract}

\tableofcontents

\chapter{Introduction}\label{sec:introduction}
\subfile{sections/introduction}

\chapter{Background}\label{sec:background}
\subfile{sections/background}

\chapter{Syntax}\label{sec:syntax}
\subfile{sections/grammar}
\section{Operational semantics}\label{sec:semantics}
\subfile{sections/semantics}

\chapter{Inference}\label{sec:inference}
\subfile{sections/inference}
\subfile{sections/inference-example}

\chapter{Rewrite}\label{sec:rewrite}
\subfile{sections/rewrite}
\subfile{sections/rewrite-example}

\chapter{Correctness}\label{sec:correctness}
\subfile{sections/correctness}
\subfile{sections/rewrite-proof}

\chapter{Implementation}\label{sec:implementation}
\subfile{sections/implementation}

\chapter{Performance}\label{sec:performance}
\subfile{sections/performance}

\chapter{Conclusion}\label{sec:conclusion}
\subfile{sections/conclusion}

\bibliographystyle{plain}
\bibliography{bibliography}

\end{document}
