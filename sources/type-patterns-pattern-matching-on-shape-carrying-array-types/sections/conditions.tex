
\section{Arbitrary constraints}

Type patterns alone are not yet sufficient for generating arbitrarily complex constraints on arguments and return values.
To preserve the benefits provided by keeping these dependencies contained to the function signature, we extend the syntax of functions by allowing them to contain any number of arbitrary expressions that operate on arguments, or on features of type patterns.
These conditions can be any built-in or user-defined boolean expression without side-effects.

In the case of the selection function, we can include two additional conditions to check whether the index vector is non-negative, and that each index in the index vector is less than the corresponding element in the shape of the array.
Ensuring that the index is always in bounds.
\begin{lstlisting}
int[d:shp]
sel(int[n] idx, int[n:outer,d:shp] a)
    | all(0 <= idx), all(idx < outer)
{
    return { iv -> a[idx ++ iv] | iv < shp };
}
\end{lstlisting}

\noindent
Because of the way we interleave constraints into the pre- and post-check functions, inserting such checks into the program requires little additional effort.
The only significant work that remains is to decide for each condition whether it is a pre-condition, or a post-condition.
If all variables that appear in a condition are defined by arguments or type patterns of arguments in a function signature, then we add a case to the pre-check function for this condition.
Otherwise, if at least one variable occurring in the condition is only defined in the type patterns of the return values, we add that condition to the post-check function instead.

As with the other checks, the condition will be inserted into the code with an if-expression and an as precise as possible error message that describes which condition failed.
The pre-check function, as previously defined in section \ref{sec:generation}, can now easily be extended to include these two new conditions:
\begin{lstlisting}[escapechar=@]
inline bool
sel_pre(int[.] idx, int[*] arr)
{
    n = shape(idx)[0];
    outer = take(n, shape(arr));
    d = dim(arr) - n;
    shp = take(d, drop(n, shape(arr)));

    pred = true;
    pred = n == dim(arr) - d ? pred
        : abort("Type pattern error ...");
    pred = @\underline{all(0 <= idx)}@ ? pred
        : abort("Type pattern error ...");
    pred = @\underline{all(idx < outer)}@ ? pred
        : abort("Type pattern error ...");
    return pred;
}
\end{lstlisting}

\noindent
We similarly extend the post-check function, for any remaining conditions that operate on features of return types.
In the case of our selection example, the post-check function remains unchanged.
Inserting the conditions into these functions allows them to potentially be statically analysed, just as the checks generated by the type patterns.
Whilst we are not able to generate an error message as precise as for type patterns, we are still able to tell the programmer which of the given conditions failed.

With the addition of these feature, our type patterns now provide a complete implementation that is strong enough to support arbitrarily complex conditions, allowing constraints on arguments and return values to be defined entirely in the function signature.
Fully separating such domain constraints from the actual logic of these functions.
