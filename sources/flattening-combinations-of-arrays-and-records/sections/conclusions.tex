
\section{Conclusion}\label{sec:conclusion}

This paper proposes an extensions of the array language \sac{} by records.
As with any other basic type in \sac{}, all records are implicitly n-dimensional arrays of records; a scalar record has dimensionality 0 and an empty shape.
As records are normal expressions, all array constructing operations of \sac{} can be applied to records as well.

The key contribution of this paper is a transformation that systematically transforms 
all arrays of records into records of arrays.
As a consequence, any non-recursive nesting of arrays and records can be transformed into records that appear as scalar records on the top level only and, thus, can be replaced by their fields entirely.
Meaning that there is no need to support records at runtime at all.
This not only reduces the implementation effort but it also avoids the overhead of nested memory allocations entirely.

The price that needs to be paid for getting rid of all records at runtime is an increase in function arguments and return values.
To counter this effect, we also propose two optimisations that identify those arguments and return values that do not contribute to the actual computations within functions that operate on records.
This reduces the increase of function signatures that results from our transformation to the necessary minimum.

To be able to perform the proposed transformation in all cases, we need to impose two restrictions.
Firstly, we have to restrict arrays in record fields to be of a fixed shape.
This restriction is a consequence of \sac{}'s limitation to homogeneously nested arrays.
In array languages without this restriction, arbitrarily shaped record fields would be directly possible, using the same transformation scheme as proposed in the paper.

Secondly, we have to restrict records to be defined in a non-recursive manner.
This is a fundamental limitation of the proposed flattening approach.
As the memory size of flattened arrays needs to be known prior to computing array values, recursively defined records cannot be flattened away as proposed in the paper.
However, for most compute intensive applications such recursion is not needed.
Language support for such recursion surely is possible and would constitute an extension of the language capabilities of \sac{}, but it would be orthogonal to the transformation work presented here.
Our current implementation in the \sac{} compiler at \url{www.sac-home.org} does not support such recursion.

While our records do not support methods as part of the records, the function overloading of \sac{} provides the programmer with the opportunity to define record-type-specific functionalities.
Even though no subtyping relation between record types is supported in \sac{}, it does support subtyping in array types~\cite{homogeneousness} in general.
Extending this to records as they are suggested in this paper might be interesting future work.
