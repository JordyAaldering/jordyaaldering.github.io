
\section{Case Study}

In section~\ref{sec:examples} we have seen an implementation of the \texttt{take} function, which is used in practice in many array programs.
Applications of this function are prone to out-of-bounds errors: the shape vector argument can have too many elements, or one of its elements can be greater than the corresponding extent in the array argument.
Following is an example of this first case, where the array argument is one-dimensional but a two-element shape argument is provided.
\begin{lstlisting}
arr = genarray ([10], 0f);
res = take ([7,9], arr);
\end{lstlisting}

\noindent
Without the use of type patterns, this invalid application of \texttt{take} produces the following compile time error:
\begin{lstlisting}[language=console]
argument #1 should not exceed the shape of
argument #2 of "_drop_SxV_"; types found:
int{2} and int[1]
\end{lstlisting}

\noindent
This error may come at a surprise as our program does not even make direct use of \verb|_drop_SxV_|.
To find out where this comes from, users can enable a stack trace. For our example, this yields:
\begin{lstlisting}[language=console,escapechar=@]
./example.sac:6: error:
   -- in Array::sel( int[2], float[.])
@\vspace{-7pt}@
./example.sac:12: error:
   -- in _MAIN::take( int[2]{7,9}, float[.])
@\vspace{-7pt}@
./example.sac:18: error:
   -- in _MAIN::foo( float[.])
\end{lstlisting}

\noindent
In this case, we can see that the failing application of \verb|_drop_SxV_| stems from a call to the selection function
in the standard library (here: \verb|Array|) which in turn was called by our version of \verb|take|. However,
it is not immediately clear that an erroneous call of \verb|take| is the culprit, as we see the entire stack trace.
In particular when dealing with long call sequences, a manual check where the unwanted function call happens can become very tedious.

In contrast, the type pattern described in this paper enable the compiler to directly point to the erroneous call.
With type patterns, we get the following pre-condition compile time error:
\begin{lstlisting}[language=console]
Type pattern error in application of take:
dimensionality of argument `arr' is too small,
could not assign a non-negative value to `m'
in `m:inner'
\end{lstlisting}

\noindent
Similarly, type patterns also improve error messages from function implementations that do not meet the co-domain constraints, even if the domain constraints are met.
For example, let us assume a problem within the definition of take, where the upper bound incorrectly uses a $\leq$ instead of $<$, i.e., we have:
\begin{lstlisting}
return { iv -> arr[iv] | iv <= shp };
\end{lstlisting}

\noindent
Furthermore, let us assume we now call \verb|take([10], arr)|.
Without type patterns, we get an error from the out of bounds selection:
\begin{lstlisting}[language=console]
Scalar constraint `SACp_eat_43 (10) <
SACp_emal_7118__uprf_2476 (10)' violated
\end{lstlisting}

\noindent
If we add type patterns to the definition of \texttt{take} we instead get a more descriptive post-condition error message, which makes it clear that there is a mistake in the definition of \texttt{take}:
\begin{lstlisting}[language=console]
Type pattern error in definition of take: the
found value of `shp' in `n:shp' of the return
value is not equal to the value of argument `shp'
\end{lstlisting}

\noindent
We even obtain this error for examples such as \verb|take([8], arr)| where this bug goes unnoticed in the variant without type patterns.
