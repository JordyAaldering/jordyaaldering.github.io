
\subsection{Using type patterns}\label{sec:examples}

Making use of these type patterns, we can extract the shape and the rank of an argument without requiring the use of built-in functions, allowing for simple re-definitions of the functions \texttt{dim} and \texttt{shape}:
\begin{lstlisting}
int                      int[d]
dim(int[d:shp] arr)      shape(int[d:shp] arr)
{                        {
    return d;                return shp;
}                        }
\end{lstlisting}

\noindent
In the example of \texttt{shape} we can see how the type pattern notation allows for an intuitive way to express the relation between the rank of the argument and the shape of the result.

Type patterns also allow us to impose constraints between multiple arguments and return values of a function.
For example, the function signature of shape-polymorphic addition of two floating point arrays can be specified by:
\begin{lstlisting}
float[d:shp] 
add(float[d:shp] a, float[d:shp] b)
{
    return { iv -> a[iv] + b[iv] };
}
\end{lstlisting}

\noindent
Here the type patterns in the signature demand both arguments to have the exact same shape.
Additionally it specifies that the shape of the result will also be equal to the shapes of those two inputs.

An example of a more intricate shape relation between argument and result shapes can be seen in a simplified definition of the \texttt{take} function:
\begin{lstlisting}
float[n:shp,m:inner]
take(int[n] shp, float[n:outer,m:inner] arr)
{
    return { iv -> arr[iv] | iv < shp };
}
\end{lstlisting}

\noindent
We see here that the result has the same rank as the second argument, where the first \texttt{n} elements of the result shape are defined through the values of the first argument.
These type patterns demonstrate an interesting case: when looking at the type pattern of argument \texttt{arr} in isolation, we have two features that match against a variable number of ranks captured by the variables \texttt{n} and \texttt{m}.
This is potentially ambiguous.
However, in the given context, we have a uniquely determined constraint for \texttt{n} through the shape of the first argument.

This potential ambiguity is inevitable as soon as we have more than one variable-rank feature within a type pattern.
Fortunately, these cases can be identified and rejected statically, during the translation of type patterns into explicit constraint checking code.

Finally we consider the shape-polymorphic selection function, which uses the \verb|_sel_VxA_| primitive (applied using \texttt{arr[$...$]}) to select a single scalar, or a larger sub-array, from a given array.
This function will serve as our running example throughout the remainder of this paper.
\begin{lstlisting}
int[d:shp]
sel(int[n] idx, int[n:outer,d:shp] arr)
{
    return { iv -> arr[idx ++ iv] | iv < shp };
}
\end{lstlisting}

\noindent
Again, we see here a potential ambiguity between the two variable-rank features of argument \texttt{arr}, captured by the variables \texttt{n} and \texttt{d}.
This ambiguity is resolved through the occurrence of \texttt{n} within the type pattern of the first argument.
