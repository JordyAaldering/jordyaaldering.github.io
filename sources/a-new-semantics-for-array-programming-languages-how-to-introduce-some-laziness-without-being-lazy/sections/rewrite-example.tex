\documentclass[../main.tex]{subfiles}
\begin{document}

\section{Example}
Lets now take a look at an example of a rewrite. We will be extending the example \texttt{take n xs} from the inference. We will use this function to implement a function that shifts the values of an array left or right by some amount \texttt{n}.

This program is shown below, along with the pre-computed demands for each user-defined function, these are computed like we did in the inference section (\ref{sec:inference}). The program creates a list ranging from $0$ to $99$ and shifts those values 20 indices to the right, placing zeros at the now empty indices.
\begin{minted}{text}
# demand: [[0,3,3,3], [0,1,2,3]]
let take = \n.\arr.
    let ofs = if n > 0 then 0
        else (shape arr).([0]) + n
    in
    gen [|n|] 0 with [n * 0] <= iv < [|n|] in
        arr.(iv + ofs)
in

# demand: [[0,3,3,3], [0,1,2,3]]
let drop = \n.\arr.
    if n > 0 then
        take (n - (shape arr).([0])) arr
    else
        take ((shape arr).([0]) + n) arr
in

# demand: [[0,3,3,3], [0,1,2,3]]
let shift = \n.\arr.
    let pad = gen (shape (take n arr)) 0 in     # (1)
    let xs = drop (-n) arr in
    if n > 0 then pad ++ xs
        else xs ++ pad
in

let size = 20000 in
let arr = gen size 0
    with 0 <= iv < size in iv in
shift 5000 arr
\end{minted}

As can be seen in the first line of the shift function (1), we compute the entire result of \texttt{take}, but then only use its shape. So instead of looking at rewriting the entire program we will look at how this line would be rewritten, as it is the most interesting.

Lets assume we want the full result of this expression and that the parameters have not been rewritten to a lower level, we would then have to apply the following rewrite rule to the with-expression:
\begin{align*}
\reduce{F}{\genexpr{(\shpexpr{(take\; n\; arr)})}{0}}
    &= \genexpr{\reduce{F}{\shpexpr{(take\; n\; arr)}}}{\reduce{F}{0}} \\
    &= \genexpr{\reduce{F}{\shpexpr{(take\; n\; arr)}}}{0}
\end{align*}
%
We know that if we want the full value of a shape, we can rewrite the inner expression to the shape level.
We would officially use currying currying here, but we won't worry about those extra steps in this example and just combine them.
\begin{align*}
\reduce{F}{\genexpr{(\shpexpr{(take\; n\; arr)})}{0}}
    &= \genexpr{\reduce{S}{(take\; n)\; arr}}{0} \\
    &= \genexpr{\reduce{S}{take\; n\; arr}}{0}
\end{align*}
%
Now we are going to need a new version of the \texttt{take} function, which we will call \texttt{take\_s}. Usually this would already been done by this point but for readability we'll change the order up a bit. First we can the let-expression by removing the variable $\textit{ofs}$ (offset), this is possible because the demand of $\textit{ofs}$ in the with-expression is $\N$. The rewrite of the with-expression is also very simple as we only need its shape, which is given by the programmer.
\begin{align*}
\reduce{S}{\letexpr{\textit{ofs}}{e_{cond}}{e_{with}}}
    &= \reduce{S}{e_{with}} \\
    &= \reduce{F}{[|n|]} \\
    &= [\reduce{F}{|n|}] \\
    &= [|\reduce{F}{n}|] \\
    &= [|n|]
\end{align*}
%
So our \texttt{take n arr} function is now a lot simpler.
\begin{align*}
    \text{let } take\_s = \lambdaexpr{n}{\lambdaexpr{arr}{[|n|]}}
\end{align*}
%
We now get the application rule, but since \texttt{take} is a pointer to a lambda expression, we will point to the rewritten function of the required level by changing the name of this pointer. We also need to know to what level we need to rewrite the variables, for this we use the demand we have found earlier. Again, in the rules we would use currying, but here we will just do both parameters at once.
\begin{align*}
\reduce{F}{\genexpr{(\shpexpr{(take\; n\; arr)})}{0}}
    &= \genexpr{(take\_s\; \reduce{X}{n}\; \reduce{Y}{arr})}{0} \\
        &\quad\text{where } \X = \pv{\texttt{take}}[0][2] \\
        &\hspace{47pt} = [[0,3,3,3], [0,1,2,3]][0][2] \\
        &\hspace{47pt} = [0,3,3,3][2] \\
        &\hspace{47pt} = 3 \\
        &\quad\text{and } \mathcal{Y} = \pv{\texttt{take}}[1][2] \\
        &\hspace{37.5pt} = [[0,3,3,3], [0,1,2,3]][1][2] \\
        &\hspace{37.5pt} = [0,1,2,3][2] \\
        &\hspace{37.5pt} = 2 \\
    &= \genexpr{(take\_s\; \reduce{F}{n}\; \reduce{S}{arr})}{0}
\end{align*}
%
Which finally results in the program below. Especially if we expand \texttt{take\_s} we see that this computation is a lot faster and skips a lot of steps that have become unnecessary when computing only the shape.
\begin{align*}
\reduce{F}{\genexpr{(\shpexpr{(take\; n\; arr)})}{0}}
    &= \genexpr{(take\_s\; n\; (\shpexpr{arr}))}{0} \\
    &= \genexpr{((\lambdaexpr{n}{\lambdaexpr{arr}{[|n|]}})\; n\; (\shpexpr{arr}))}{0} \\
\end{align*}

\end{document}
